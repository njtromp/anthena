%----------------------------------------------------------------------------------------
% LATEX LABBOOK TEMPLATE
% Versie 1.0 (12 september 2013)
% Opmerkingen of feedback naar Robert van Wijk 
%         (robertvanwijk@uva.nl)
%----------------------------------------------------------------------------------------

%----------------------------------------------------------------------------------------
% PACKAGES EN DOCUMENT CONFIGURATIE
%----------------------------------------------------------------------------------------

\documentclass[a4paper,12pt]{article}
\usepackage{graphicx}
\usepackage{fancyhdr}
\usepackage{lastpage}
\usepackage{xifthen}
\usepackage{algorithm2e}
\usepackage{lipsum}
\usepackage{float}
\usepackage{hyperref}
\usepackage{gensymb}

%----------------------------------------------------------------------------------------
% HEADER & FOOTER
%----------------------------------------------------------------------------------------
\pagestyle{fancy}
  \lhead{\includegraphics[width=7cm]{logoUvA}}    %Zorg dat het logo in dezelfde map staat
  \rhead{\footnotesize \textsc {Technical Report\\ \opdracht}}
  \lfoot
    {
  \footnotesize \studentA
  \ifthenelse{\isundefined{\studentB}}{}{\\ \studentB}
  \ifthenelse{\isundefined{\studentC}}{}{\\ \studentC}
  \ifthenelse{\isundefined{\studentD}}{}{\\ \studentD}
  \ifthenelse{\isundefined{\studentE}}{}{\\ \studentE}
    }
  \cfoot{}
  \rfoot{\small \textsc {Page \thepage\ of \pageref{LastPage}}}
  \renewcommand{\footrulewidth}{0.5pt}

\fancypagestyle{firststyle}
 {
  \fancyhf{}
   \renewcommand{\headrulewidth}{0pt}
   \chead{\includegraphics[width=7cm]{logoUvA}}
   \rfoot{\small \textsc {Page \thepage\ of \pageref{LastPage}}}
 }

\setlength{\topmargin}{-0.3in}
\setlength{\textheight}{630pt}
\setlength{\headsep}{40pt}

%----------------------------------------------------------------------------------------
% DOCUMENT INFORMATIE
%----------------------------------------------------------------------------------------
%Geef bij ieder command het juiste argument voor deze opdracht. Vul het hier in en het komt op meerdere plekken in het document correct te staan.

\newcommand{\titel}{Athena Line Follower}   %Zelfbedachte titel
\newcommand{\opdracht}{Practical Project}   %Naam van opdracht die je van docent gehad hebt
\newcommand{\docent}{Sebastian Altmeyer}
\newcommand{\cursus}{Embedded Software \& Systems}
\newcommand{\vakcode}{5364EMSS6Y}   %Te vinden op oa Datanose
\newcommand{\datum}{\today}         %Pas aan als je niet de datum van vanaag wilt hebben
\newcommand{\studentA}{Rocco Andela}
\newcommand{\uvanetidA}{11745673}
\newcommand{\studentB}{Anna Valachi}      %Comment de regel als je allen werkt
\newcommand{\uvanetidB}{12301922}
\newcommand{\studentC}{Nico Tromp}    %Uncomment de regel als je met drie studenten werkt
\newcommand{\uvanetidC}{11699353}
\newcommand{\studentD}{Sjoerd van der Heijden}    %Uncomment de regel als je met vier studenten werkt
\newcommand{\uvanetidD}{10336001}
\newcommand{\studentE}{Vasilis Gemistos}      %Uncomment de regel als je met vijf studenten werkt
\newcommand{\uvanetidE}{12318264}

%----------------------------------------------------------------------------------------
% AUTOMATISCHE TITEL
%----------------------------------------------------------------------------------------
\begin{document}
\thispagestyle{firststyle}
\begin{center}
  \textsc{\Large \opdracht}\\[0.2cm]
    \rule{\linewidth}{0.5pt} \\[0.4cm]
      { \huge \bfseries \titel}
    \rule{\linewidth}{0.5pt} \\[0.2cm]
  {\large \datum  \\[0.4cm]}
  
  \begin{minipage}{0.4\textwidth}
    \begin{flushleft} 
      \emph{Students:}\\
      {\studentA \\ {\small \uvanetidA \\[0.2cm]}}
        \ifthenelse{\isundefined{\studentB}}{}{\studentB \\ {\small \uvanetidB \\[0.2cm]}}
        \ifthenelse{\isundefined{\studentC}}{}{\studentC \\ {\small \uvanetidC \\[0.2cm]}}
        \ifthenelse{\isundefined{\studentD}}{}{\studentD \\ {\small \uvanetidD \\[0.2cm]}}
        \ifthenelse{\isundefined{\studentE}}{}{\studentE \\ {\small \uvanetidE \\[0.2cm]}}
    \end{flushleft}
  \end{minipage}
~
  \begin{minipage}{0.4\textwidth}
    \begin{flushright} 
      \emph{Coordinator:} \\
      \docent \\[0.2cm]
      \emph{Course:} \\
      \cursus \\[0.2cm]
      \emph{Course number:} \\
      \vakcode \\[0.2cm]
    \end{flushright}
  \end{minipage}\\[1 cm]
\end{center}

\newpage

\section{Introduction - The fight for a new dawn}
The year is 2018. Mankind is at the brink of extinction. Earths resources are almost depleted. The only way to survive is to colonize other planets. A candidate planet named Traal has been found. The best five engineers in the world have been put together to build a rover to explore the new planet.\\

\noindent The name of the courageous robot is Athena. Athena was a patron goddess associated with wisdom, handicraft, and warfare. These are the qualities that are need in an explorer that wanders the dangerous planes of the planet Traal.\\

\noindent The ultimate goal of Athena is to find a field of resources. However the planet is inhabited by the Ravenous Bugblatter Beast. The five engineers decided not to exterminate the whole population of Traal so they do not equip Athena with weapons. The only option, when the beast is encountered, is to flee.

\section{Contributions}
The engineers came up with a specific plan. The plan consists of some steps. First, the rover is manually driven to the starting position. Then autonomous mode is engaged. The robot drives forward until it encounters the Ravenous Bugblatter Beast. It has to stay at an safe distance of 10-12cm from the beast, beep for half a second and after 5 seconds it has to flee at a 90\degree angle to the right until it finds the black line that leads to the green field of resources. There it has to raise a flag and phone home to earth.\\

\noindent The engineers explored many different solutions before visiting the Traal Planet. They wanted to be sure that they were able to perform the tasks in the best possible way.
\subsection{Who did what}
Each of the engineers has their own field of expertise.\\
\begin{itemize}
    \item Engineer Vasilis is an excellent communications expert. He was assigned the tasks of phoning home and raising the flag when the rover has found the resources. Whenever there is an encounter with a beast his work makes sure that Athena flees in the right direction.
    \item Engineer Anna worked one of the solutions for following the black line and made a library for the color sensor.
    \item Engineer Sjoerd took up the responsibility to combine all of the work of the other engineers in one complete package.
    \item Engineer Nico worked on a second solution for following the black line. He was also responsible for making sure that the robot would drive completely straight in both the autonomous and manual mode.
    \item Engineer Rocco worked on a third approach for following the black line. He was also responsible for the manual controls of the robot.
\end{itemize}
\subsection{Grade distribution}
All of the engineers contributed equally in the project. 
\section{Motivation of model}


\bibliographystyle{acm}
\bibliography{test.bib}

\end{document}
